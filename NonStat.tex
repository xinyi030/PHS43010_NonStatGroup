% Options for packages loaded elsewhere
\PassOptionsToPackage{unicode}{hyperref}
\PassOptionsToPackage{hyphens}{url}
%
\documentclass[
]{article}
\usepackage{amsmath,amssymb}
\usepackage{iftex}
\ifPDFTeX
  \usepackage[T1]{fontenc}
  \usepackage[utf8]{inputenc}
  \usepackage{textcomp} % provide euro and other symbols
\else % if luatex or xetex
  \usepackage{unicode-math} % this also loads fontspec
  \defaultfontfeatures{Scale=MatchLowercase}
  \defaultfontfeatures[\rmfamily]{Ligatures=TeX,Scale=1}
\fi
\usepackage{lmodern}
\ifPDFTeX\else
  % xetex/luatex font selection
\fi
% Use upquote if available, for straight quotes in verbatim environments
\IfFileExists{upquote.sty}{\usepackage{upquote}}{}
\IfFileExists{microtype.sty}{% use microtype if available
  \usepackage[]{microtype}
  \UseMicrotypeSet[protrusion]{basicmath} % disable protrusion for tt fonts
}{}
\makeatletter
\@ifundefined{KOMAClassName}{% if non-KOMA class
  \IfFileExists{parskip.sty}{%
    \usepackage{parskip}
  }{% else
    \setlength{\parindent}{0pt}
    \setlength{\parskip}{6pt plus 2pt minus 1pt}}
}{% if KOMA class
  \KOMAoptions{parskip=half}}
\makeatother
\usepackage{xcolor}
\usepackage[margin=1in]{geometry}
\usepackage{color}
\usepackage{fancyvrb}
\newcommand{\VerbBar}{|}
\newcommand{\VERB}{\Verb[commandchars=\\\{\}]}
\DefineVerbatimEnvironment{Highlighting}{Verbatim}{commandchars=\\\{\}}
% Add ',fontsize=\small' for more characters per line
\usepackage{framed}
\definecolor{shadecolor}{RGB}{248,248,248}
\newenvironment{Shaded}{\begin{snugshade}}{\end{snugshade}}
\newcommand{\AlertTok}[1]{\textcolor[rgb]{0.94,0.16,0.16}{#1}}
\newcommand{\AnnotationTok}[1]{\textcolor[rgb]{0.56,0.35,0.01}{\textbf{\textit{#1}}}}
\newcommand{\AttributeTok}[1]{\textcolor[rgb]{0.77,0.63,0.00}{#1}}
\newcommand{\BaseNTok}[1]{\textcolor[rgb]{0.00,0.00,0.81}{#1}}
\newcommand{\BuiltInTok}[1]{#1}
\newcommand{\CharTok}[1]{\textcolor[rgb]{0.31,0.60,0.02}{#1}}
\newcommand{\CommentTok}[1]{\textcolor[rgb]{0.56,0.35,0.01}{\textit{#1}}}
\newcommand{\CommentVarTok}[1]{\textcolor[rgb]{0.56,0.35,0.01}{\textbf{\textit{#1}}}}
\newcommand{\ConstantTok}[1]{\textcolor[rgb]{0.00,0.00,0.00}{#1}}
\newcommand{\ControlFlowTok}[1]{\textcolor[rgb]{0.13,0.29,0.53}{\textbf{#1}}}
\newcommand{\DataTypeTok}[1]{\textcolor[rgb]{0.13,0.29,0.53}{#1}}
\newcommand{\DecValTok}[1]{\textcolor[rgb]{0.00,0.00,0.81}{#1}}
\newcommand{\DocumentationTok}[1]{\textcolor[rgb]{0.56,0.35,0.01}{\textbf{\textit{#1}}}}
\newcommand{\ErrorTok}[1]{\textcolor[rgb]{0.64,0.00,0.00}{\textbf{#1}}}
\newcommand{\ExtensionTok}[1]{#1}
\newcommand{\FloatTok}[1]{\textcolor[rgb]{0.00,0.00,0.81}{#1}}
\newcommand{\FunctionTok}[1]{\textcolor[rgb]{0.00,0.00,0.00}{#1}}
\newcommand{\ImportTok}[1]{#1}
\newcommand{\InformationTok}[1]{\textcolor[rgb]{0.56,0.35,0.01}{\textbf{\textit{#1}}}}
\newcommand{\KeywordTok}[1]{\textcolor[rgb]{0.13,0.29,0.53}{\textbf{#1}}}
\newcommand{\NormalTok}[1]{#1}
\newcommand{\OperatorTok}[1]{\textcolor[rgb]{0.81,0.36,0.00}{\textbf{#1}}}
\newcommand{\OtherTok}[1]{\textcolor[rgb]{0.56,0.35,0.01}{#1}}
\newcommand{\PreprocessorTok}[1]{\textcolor[rgb]{0.56,0.35,0.01}{\textit{#1}}}
\newcommand{\RegionMarkerTok}[1]{#1}
\newcommand{\SpecialCharTok}[1]{\textcolor[rgb]{0.00,0.00,0.00}{#1}}
\newcommand{\SpecialStringTok}[1]{\textcolor[rgb]{0.31,0.60,0.02}{#1}}
\newcommand{\StringTok}[1]{\textcolor[rgb]{0.31,0.60,0.02}{#1}}
\newcommand{\VariableTok}[1]{\textcolor[rgb]{0.00,0.00,0.00}{#1}}
\newcommand{\VerbatimStringTok}[1]{\textcolor[rgb]{0.31,0.60,0.02}{#1}}
\newcommand{\WarningTok}[1]{\textcolor[rgb]{0.56,0.35,0.01}{\textbf{\textit{#1}}}}
\usepackage{graphicx}
\makeatletter
\def\maxwidth{\ifdim\Gin@nat@width>\linewidth\linewidth\else\Gin@nat@width\fi}
\def\maxheight{\ifdim\Gin@nat@height>\textheight\textheight\else\Gin@nat@height\fi}
\makeatother
% Scale images if necessary, so that they will not overflow the page
% margins by default, and it is still possible to overwrite the defaults
% using explicit options in \includegraphics[width, height, ...]{}
\setkeys{Gin}{width=\maxwidth,height=\maxheight,keepaspectratio}
% Set default figure placement to htbp
\makeatletter
\def\fps@figure{htbp}
\makeatother
\setlength{\emergencystretch}{3em} % prevent overfull lines
\providecommand{\tightlist}{%
  \setlength{\itemsep}{0pt}\setlength{\parskip}{0pt}}
\setcounter{secnumdepth}{-\maxdimen} % remove section numbering
\newlength{\cslhangindent}
\setlength{\cslhangindent}{1.5em}
\newlength{\csllabelwidth}
\setlength{\csllabelwidth}{3em}
\newlength{\cslentryspacingunit} % times entry-spacing
\setlength{\cslentryspacingunit}{\parskip}
\newenvironment{CSLReferences}[2] % #1 hanging-ident, #2 entry spacing
 {% don't indent paragraphs
  \setlength{\parindent}{0pt}
  % turn on hanging indent if param 1 is 1
  \ifodd #1
  \let\oldpar\par
  \def\par{\hangindent=\cslhangindent\oldpar}
  \fi
  % set entry spacing
  \setlength{\parskip}{#2\cslentryspacingunit}
 }%
 {}
\usepackage{calc}
\newcommand{\CSLBlock}[1]{#1\hfill\break}
\newcommand{\CSLLeftMargin}[1]{\parbox[t]{\csllabelwidth}{#1}}
\newcommand{\CSLRightInline}[1]{\parbox[t]{\linewidth - \csllabelwidth}{#1}\break}
\newcommand{\CSLIndent}[1]{\hspace{\cslhangindent}#1}
\usepackage{hyperref}
\hypersetup{colorlinks=true, citecolor=blue, linkcolor=blue, urlcolor=blue}
\usepackage{hyperref}
\ifLuaTeX
  \usepackage{selnolig}  % disable illegal ligatures
\fi
\IfFileExists{bookmark.sty}{\usepackage{bookmark}}{\usepackage{hyperref}}
\IfFileExists{xurl.sty}{\usepackage{xurl}}{} % add URL line breaks if available
\urlstyle{same}
\hypersetup{
  pdftitle={Final Project: Bayesian Inference Clinical Trials and Nonparametric Models},
  pdfauthor={Non-stat Team: Captain: Xinyi Zhang, Members: \ldots{}},
  hidelinks,
  pdfcreator={LaTeX via pandoc}}

\title{Final Project: Bayesian Inference Clinical Trials and
Nonparametric Models}
\author{Non-stat Team: Captain: Xinyi Zhang, Members: \ldots{}}
\date{2023-05-22}

\begin{document}
\maketitle

\hypertarget{introduction-by-hongzhang-xie}{%
\section{1. Introduction (by Hongzhang
Xie)}\label{introduction-by-hongzhang-xie}}

Phase I dose-finding clinical trials are conducted to analyze the safety
and tolerability of a new drug. This is a critical step in drug
development as it marks the first time the new drug is used in humans.
The trials aim to determine the Maximum Tolerated Dose (MTD) and give a
foundation for further drug development and clinical trials.

Because we are not fully understood the actions and side effects of new
drugs in the human body, there is a potential risk to the trial
participants. Also, experimenting with new drugs at low doses level may
delay patients' chances of treatment. On the other hand, Phase I trials
typically require close health monitoring of participants, which
requires substantial medical resources and specialized personnel.
Therefore, considering ethical and cost factors, phase I dose-finding
trials often have a small sample size, with each dose-level cohort
comprising only around three participants.

\hypertarget{why-frequency-is-not-suitable-for-phase-i-dose-finding-trials}{%
\subsection{1.1 Why frequency is not suitable for phase I dose-finding
trials?}\label{why-frequency-is-not-suitable-for-phase-i-dose-finding-trials}}

The traditional frequency method presumes a fixed coefficient for the
Dose Limiting Toxicity (DLT) distribution under the same dose level.
Each sample point is an independent draw. Given a large enough sample
size, the sampling mean will finally equal the actual coefficient.
Hence, we can make a hypothesis test for finding the MTD (such as the
assumption that DLT at a particular dose level smaller than 0.33). If we
observe enough extreme data based on the null hypothesis, we can reject
this hypothesis. However, as the frequency hypothesis test is based on
the law of large numbers, there will be a huge variance in phase I
dose-finding trials because of the small sample size. In such cases, the
likelihood of a type II error (accepting an incorrect null hypothesis)
would be very high. The results of the hypothesis test are not robust.

On the other hand, in the process of drug research and development, we
often have some prior information about the drug (such as the results of
animal experiments). This information is of great help in determining
the MTD. However, the frequency method cannot make good use of this
information.

Finally, the conclusion of the hypothesis test only rejected or failed
to reject. Even if we reject the null hypothesis (e.g.~H0: DLT
\textless{} TTL), the hypothesis test still says nothing to the H1. In
other words, rejecting the null hypothesis (H0) just means that the
probability of observing the current data is extremely small when H0 is
true. It does not give us any evidence for H1 (e.g.~DLT \textgreater=
TTL). We do not know the degree of confidence for DLT \textgreater= TTL.
But in phase I dose-finding trials, clinicians need this information to
decide the following trial strategy. Therefore, it is not suitable for
the frequency method in phase I dose-finding trials.

\hypertarget{traditional-33-method}{%
\subsection{1.2 Traditional 3+3 method}\label{traditional-33-method}}

Currently, traditional 3+3 method is the mainstream method for
determining MTD in phase I dose-finding trials. By the literature review
from Tongtong Jin, more than 90\% of phase I dose-finding trials use
rule-based designs such as traditional 3+3.

The 3+3 method is intuitive, easy to operate, and provides a clear and
repeatable method to confirm the MTD, which can be more easily
understood and accepted by clinical researchers, approval agencies, and
ethics committees with non-statistical backgrounds. At the same time,
the 3+3 method is cautious about increasing the dose level and has been
tested by many previous studies, which shows that 3+3 has good safety.

However, a large number of studies have pointed out that the 3+3 method
also has some disadvantages.

\begin{enumerate}
\def\labelenumi{\alph{enumi}.}
\item
  The 3+3 method cannot efficiently utilize all the information in the
  trials. The decision at each dose level depends only on the
  information of the three (or six) experimental participants at the
  current dose level. The 3+3 method does not consider the information
  outside this dose level.
\item
  In the process of increasing the dose level, at least three subjects
  are required for each dose level, and the clinicians should wait for
  the result in the current dose level to make the decision to increase
  the dose level. Even if the current results show that the preset
  measurement increases too slowly, it is hard to adjust the dose level
  during the experiment according to the new information. This may
  expose more patients to ineffective low doses level treatment and may
  also delay the overall trial period. (This also shows that the 3+3
  method lacks flexibility).
\item
  When determining MTD, the percentage of DLT is fixed at 33\%. The TTR
  (Target toxicity rate) cannot be set flexibly.
\item
  Unable to establish a dose-response curve: The 3+3 method can only
  find a fixed MTD, but cannot build a model to show the relationship of
  the toxicity in different dose rate.
\end{enumerate}

\hypertarget{crm-method}{%
\subsection{1.3 CRM method}\label{crm-method}}

After each trial in a certain dose level, CRM will update the model
based on all the data (posterior) and then use this updated model to
predict the toxicity risk of each dose level. Based on these results,
the clinicians will decide on the next dose level and continue the trial
until meeting the stopping rules. Compared with the traditional 3+3
method, CRM has the following advantages:

\begin{enumerate}
\def\labelenumi{\alph{enumi}.}
\item
  Compared with 3+3 methods, CRM makes better use of all the data that
  has been collected during the phase I dose-finding trials. In other
  words, CRM is not only considered the information in the current dose
  level (3+3) but also considers the DLT information of all previous
  doses. So CRM can estimate MTD more accurately.
\item
  In the process of phase I dose-finding trials, CRM can easily adjust
  the subsequent test design and dose level according to the existing
  experimental information. Therefore, CRM can reduce the sample size
  and time wasted during the low-dose level (or reduce the risk of
  severe toxic reactions by reducing the dose level in the next trial),
  thereby reducing the risk of patients' exposure to extremely high or
  low dose levels (high DLT risk or the risk of delayed treatment).
\item
  CRM can analyze the complex relationship between dose and toxicity by
  constructing different Dose-toxicity skeleton models and can adapt to
  different situations by changing the model's parameters. For example,
  suppose there are two drugs, the DLT probability of the first drug is
  linearly related to the dose level. The DLT probability of the second
  drug is extremely low at low dose levels, and the DLT probability
  increases rapidly after a certain dose level. In the 3+3 method, the
  strategies of the two drugs are exactly the same, while in the CRM, we
  can use the power model for the first drug and the logistic model for
  the second drug for a better fit of the model.
\item
  The CRM method uses the Bayesian method to construct a dose-response
  curve, which can better reflect the relationship between toxicity and
  dose, and can be used to estimate the MTD at any TTR level.
\end{enumerate}

\hypertarget{why-is-our-final-project-important}{%
\subsection{1.4 Why is our final project
important?}\label{why-is-our-final-project-important}}

Although the continuous reassessment approach (CRM) has many advantages,
it also has some potential disadvantages and limitations:

\begin{enumerate}
\def\labelenumi{\alph{enumi}.}
\item
  CRM is a relatively complicated method, and there is no mature SOP
  (standard operating procedure). This means that the experiment needs a
  statistical expert for inference and decision-making, and the model
  may be too complex for the clinical team and the SMC (Safety
  Monitoring Committee or Data and Safety Monitoring Board) member. This
  has become one of the main barriers to the uptake of CRM methods in
  phase I dose-finding trials.
\item
  The dose level in the CRM method increases rapidly under some models
  (especially single-parameter models), which may increase the risk of
  the number and severity of DLT while improving the efficiency of the
  experiment. For this reason, some ad-hoc rules may be needed to
  control the risk of overdose.
\item
  The implementation of CRM requires the clinical team to choose
  appropriate models and parameters. Improper selection or setting may
  affect the results of CRM. Especially in the early stages of the
  trial, due to the lack of data, the results of CRM may have large
  fluctuations in the different prior models.
\end{enumerate}

It should be noted that the development of a drug often has a large
amount of research investment and potential marketing profits. Due to
the high flexibility of CRM, the clinical team may modify the model base
on their personal profits but not based on statistics and medical
theory, even though this may put patients at a higher risk of DLT.
Therefore, we need to further analyze the effect of models and
parameters setting on CRM inference and then establish a complete SOP
and avoid the problem of that events.

In our final project, we first conducted a literature review on Wheeler
et al.~l (2019) and twenty relevant studies. Subsequently, we use the
R-studio to make simulations to find the difference between two
scenarios with different true toxicity probabilities ((0.25, 0.3, 0.5,
0.6, 0.7) vs (0.01, 0.05, 0.2, 0.3, 0.5)) on the estimated MTD in the
CRM model. Finally, we discussed why this difference happens and how to
avoid this problem.

We believe that our final project research can help us better understand
the impact of toxicity probabilities difference on the MTD results in
the CRM model. This finding can help the clinical team estimate MTD more
accurately and helps in making the mature SOP of CRM in phase I
dose-finding trials.

\newpage

\hypertarget{literature-review}{%
\section{2. Literature Review}\label{literature-review}}

\hypertarget{crm-design-for-phase-1-dose-finding-trials-by-william}{%
\subsection{2.1 CRM design for phase 1 dose-finding trials (By
William)}\label{crm-design-for-phase-1-dose-finding-trials-by-william}}

CRM is an alternative to the standard 3+3 design based on using a model
like a one parameter or two parameter logistic model, to understand the
maximum tolerated dose in a phase 1 trial. CRM is more accurate in
choosing the MTD, is less likely to choose ineffective doses, treats
fewer patients at overly toxic doses, and treats fewer patients at very
low doses.(Garrett-Mayer 2006) Our paper plans to look at a
two-parameter model. A two-parameter model is likely to better estimate
the shape of the entire dose-toxicity relationship, (Chevret 1993) but
less efficiently identify the MTD; it may take longer to reach the MTD
since two parameters must be estimated, and there may be difficulties
fitting the model or obtaining consistent estimates of model parameters.
(Iasonos et al. 2016)

The idea behind the CRM starts with a priori dose toxicity curve and a
chosen target toxicity rate. This curve will be refit after every cohort
(1-3 patients) toxicity outcome, is observed. At every new dose or same
dose, the all-prior data is used to update the model/curve
(Garrett-Mayer E). As required a discussion needs to take place with all
relevant stakeholders. The target toxicity level is typically set
between 20 to 25\% and can be as high as 40\%. (Brock et al. 2017;
Møller 1995) In a review of 197 phase I trials published between 1997
and 2008, the median number of dose levels explored was five (range
2--12). (Penel and Kramar 2012)

Inference or decisions can be made using a likelihood or Bayesian
methods using the accruing trial data and clinical judgment. In a
Bayesian method data from patients in the trial is used to update prior
on the model distribution which then gives a posterior distribution for
the model parameters and therefore posterior beliefs for the probability
of DLT at each dose. These posterior probabilities are used to make dose
escalation decisions. By assessing a design's operating characteristics
with a specific prior in a variety of scenarios, the prior distribution
can be recalibrated until the model makes recommendations for dose
escalations and the MTD that the trial team are happy with (Wheeler et
al. 2019).

Possible decision rules include choosing the dose with an estimated
probability of DLT closest to the TTL or, more conservatively, choosing
the dose with an estimated probability of DLT closest to, but not
greater than, the TTL. The first option allows quicker escalation
towards the true MTD but may expose more patients to overdoses. The
second option reduces the chance of overdosing patients, but may take
longer to escalate towards the true MTD (Wheeler et al. 2019)

Samples sizes are determined by the study and how and where its being
conducted. Specifying a lower bound based on Cheung's work and practical
upper bound in trial protocols. Kuen Cheung (2013) proposed formulae
that use a target average percentage of correctly selecting the MTD
(say, 50\% of the time) to obtain a lower bound for the trial sample
size (Wheeler,M Graham). Although CRM designs, like standard ones, can
halt after only 10--14 subjects, it is typically necessary to plan for
at least 18--24 total subjects, after which the probability of a correct
MTD choice rises slowly with sample size (Garrett-Mayer 2006)

Cohort size at each dose level typically is more than 1. A cohort size
of one allows better understanding of operating characteristics but this
is rarely used. (Garrett-Mayer 2006) There could be regulatory
constraints. If cohort size is greater than 2, then a monitoring plan is
needed.

Stopping rules for the trial include the following examples. Early
termination can be considered if the MTD is judged to be outside the
planned set of doses. Adding additional patients is unlikely to yield
information that would change the current MTD estimate. Fixed no of
patients has been consecutively dosed at one dose level. Estimated
probability of all dose levels having a DLT rate about the TTL is at
least 90\%. The probability that the next m patients to be dose in the
trial will be given the same dose levels, regardless of DLT outcomes
observed, exceed some level (Wheeler et al. 2019).

\hypertarget{summary-for-references}{%
\subsection{2.2 Summary for references}\label{summary-for-references}}

\hypertarget{references-1-5-by-tongtong-jin}{%
\subsubsection{2.2.1 References 1-5 (by Tongtong
Jin)}\label{references-1-5-by-tongtong-jin}}

\hypertarget{phase-1-trial-and-maximum-tolerated-dose-mtd}{%
\paragraph{Phase 1 trial and maximum tolerated dose
(MTD)}\label{phase-1-trial-and-maximum-tolerated-dose-mtd}}

The phase 1 trials in oncology are usually designed to obtain the
optimal dose of a new treatment for efficacy testing in subsequent phase
2 trials. For cytotoxic agents, the probability of treatment benefit is
presumed to be positively proportional to the dose in a certain range of
consideration. Thus, the optimal dose in phase 1 trial is usually
considered as the highest dose at a tolerable level of toxicity. And the
optimal dose we are seeking for is exactly the maximum tolerated dose
(MTD).

To define MTD in a more rigorous way, it is the dose expected to produce
some degree of medically unacceptable, dose limiting toxicity (DLT) in a
specified proportion \(\theta\) of patients. Namely,

\[Prob(DLT | Dose = MTD) = \theta,\] where the proportion \(\theta\) is
also defined as the target toxicity level (TTL) (Babb and Rogatko 2004;
Cater 1972).

\hypertarget{dose-escalation-methods-in-phase-1-trials}{%
\paragraph{Dose escalation methods in phase 1
trials}\label{dose-escalation-methods-in-phase-1-trials}}

To find the MTD we defined above, in clinical testing we adopt dose
escalation methods, which is based on the prior belief that the toxicity
increases monotonically with increasing dose. The principle of dose
escalation in phase 1 trials is both maintaining the toxicity at a safe
level and the information accumulation at a rapid speed and at the same
time avoid patients being exposed to subtherapeutic doses as much as
possible.

Dose escalation methods can be mainly classified in two branches,
rule-based designs such as traditional 3+3 design and model-based
designs such as continual reassessment method. Rule-based designs don't
make any assumptions for the function of toxicity with respect to dose
level. And the next step of dose is purely dependent on the information
from the last dose. Then finally terminates at some certain stopping
criteria. But model-based designs assume there's a specific function
between dose and toxicity, usually power functions, logistic functions
etc., and then apply accumulated information from every dose to
determine the next dose.

From the perspective of practical use, rule-based designs like
traditional 3+3 are easier to implement, but model-based designs need
biostatistical expertise and available software on site to perform
real-time model fitting. As for the information utilization, rule-based
designs only use current information, but model-based designs make use
of all toxicity information accumulated during the trial, which can be
more comprehensive. In the aspect of the exposure to subtherapeutic
doses, model-based designs relatively treat fewer patients at suboptimal
doses than rule-based ones. Hence according to the principle of dose
escalation methods, the model-based designs usually do better in rapid
information accumulation and reducing excessive exposure to
subtherapeutic doses.(Le Tourneau, Lee, and Siu 2009).

\hypertarget{current-popularity-of-rule-based-designs-and-model-based-designs-in-phase-1-trials}{%
\paragraph{Current popularity of rule-based designs and model-based
designs in phase 1
trials}\label{current-popularity-of-rule-based-designs-and-model-based-designs-in-phase-1-trials}}

Although the model-based designs show great advantages in many aspects,
rule-based designs like the 3+3 design are still more prevailingly used,
and model-based designs are rarely used. Some statistical results about
the popularity of these two types of methods in phase 1 trials are as
follows.

Rogatko et al. (2007) examined through the records of cancer phase 1
trial from the Science Citation Index database from 1991 to 2006 and
divided them into two sets (dose-finding trials and methodologic studies
of dose-escalation designs). Then track among these two sets which
trials adopted new statistical designs. As a result, only 1.6\% trials
follow one of the methodologic studies and show extensive lags on
publication time. The rest of the trials all follow the traditional
up-and-down method (a type of rule-based method).

Chiuzan et al. (2017) studied the degree of adoption of the methods with
new trial designs on early phase trials of molecularly targeted agents
(MTA) and immunotherapies. It searched papers published from 2008 to
2014 about phase 1 oncology trials and found that in dose-finding
trials, 92.9\% of them utilized rule-based designs and 5.4\% used
model-based designs or other novel designs. Particularly, among the MTA
and immunotherapies trials, 5.8\% used model-based designs. The results
show that the adoption of model-based designs and novel designs remains
low.

Above phenomenon could be caused by limited time and effort of
clinicians and statisticians and the lack of comprehensive and detailed
tutorials and instructions for the newly designed approaches.(Rogatko et
al. 2007; Chiuzan et al. 2017).

\hypertarget{references-6-10-by-yiwei-ding}{%
\subsubsection{2.2.2 References 6-10 (by Yiwei
Ding)}\label{references-6-10-by-yiwei-ding}}

\hypertarget{reference-6}{%
\paragraph{Reference 6:}\label{reference-6}}

O'Quigley and Zohar (2006)

For the atistical design of dose-finding studies, the standard design is
a `memoryless' design and it's not so satisfying. This paper describes
designs with memory and we discuss how these designs are superior to
memoryless designs. The most well-known design with memory is the
continual reassessment method (CRM).

\hypertarget{reference-7}{%
\paragraph{Reference 7:}\label{reference-7}}

Harrington et al. (2013)

Carrying out dual-agent phase I trials for medications is crucial. There
are predominantly two kinds of dose-escalation trials: rule-based and
model-based. Trials based on models are progressively adjusted with the
help of Bayesian techniques, which merge preliminary data concerning the
dose-toxicity relationship. Studies using simulations indicate that
model-driven designs tend to treat a greater proportion of patients at
near-optimal dose levels.

\hypertarget{reference-8}{%
\paragraph{Reference 8:}\label{reference-8}}

Adaptive Designs Working Group of the MRC Network of Hubs for Trials
Methodology Research (2017)

This paper summarize why modelbased designs such as the continual
reassessment method (CRM) are more available than 3+3 and similar
rulebased A+B designs. Compared with rule-based designs, Modelbased
designs can clearly define and can flexibly choose target DLT rate; many
patients can be treated at the optimal dose; few patients can be treated
at subtherapeutic doses; the utilization of available data is efficient;
extension to more complex questions is smooth \& straightforward;
deviations from the plan are easily accommodated.

\hypertarget{reference-9}{%
\paragraph{Reference 9:}\label{reference-9}}

Jaki, Clive, and Weir (2013)

There are three classes of dose-escalation trial design: gorithmic
approaches (including the popular 3+3 design), Bayesian model-based
designs and Bayesian curve-free methods. The main benefit of algorithmic
approaches is the simplicity. Model-based and curve-free Bayesian
approaches are more preferable because they are more able to identify
the dose with the desired toxicity rate and allocate a greater
proportion of patient. For statistical and practical reasons, Bayesian
model-based or curve-free approach is better. If there is sufficient
evidence of high enough quality from previous studies, the model-based
approach will be better, otherwise curve-free one is better.

\hypertarget{reference-10}{%
\paragraph{Reference 10:}\label{reference-10}}

O'Quigley, Pepe, and Fisher (1990)

For the design and analysis of Phase I clinical trails in cancer,
attention focuses rather on identifying a dose with a given targeted
level is the best estimate of this level. Such sequential designs is
called continual reassessment method (CRM). In the procedure, we update
our notion of the dose-reponse relationship. From the simulations, this
method is good.

\hypertarget{references-11-15-by-ujjwal-sehrawat}{%
\subsubsection{2.2.3 References 11-15 (by Ujjwal
Sehrawat)}\label{references-11-15-by-ujjwal-sehrawat}}

Clinicians are interested in phase 1 dose finding designs that can
estimate the MTD using fewer patients with a fixed number of doses, or
can test more dose levels for a given sample size. Several studies have
compared CRM to the traditional SM (3+3 standard method that escalates
doses after 3 patients with an option for an additional 3 patients; also
called traditional method TM or the ``up-and-down scheme) {[}D*{]} and
found that CRM is more likely to recommend the correct MTD and dose more
trial patients close to the MTD. (O'Quigley 1999; Thall and Lee 2003;
Iasonos et al. 2008; Onar, Kocak, and Boyett 2009; Onar-Thomas and Xiong
2010) We expound on some of these prior works below.

O'Quigley (1999) in response to Korn et al. (1994)'s findings about
CRM's slower time duration and poorer safety standards relative to SM
concluded that, indeed, CRM does not take longer than SM when the
comparison fairly accounts for comparable grouping inclusions and that
CRM is, indeed, safer for a randomly chosen patient than SM as per
simulation results that show that the probability of being treated at
very high toxic levels is almost always higher with SM than CRM.
Furthermore, O'Quigley et al.~argued that if treating patients at
unacceptably low sub-therapeutic levels is considered part of the safety
definition, then CRM also performs much better than SM. Unlike SM, it is
entirely straightforward to adjust CRM to make it safe as we require.
All it requires is to change the target level, say from 0.2 to 0.1. In
this case, the observed number of toxicities will be, on average,
roughly halved. One of the main advantages of CRM, O'Quigley et
al.~argued, is its flexibility and ability to be adapted to potentially
different situations unlike SM which is rigid, samples independently of
any targeted percentile, and has no convergence properties. (O'Quigley
1999)

Thall and Lee (2003) described and compared 2 practical outcome-adaptive
statistical methods for dose finding in phase 1 clinical trials: CRM and
a logistic regression model-based method. Both methods used Bayesian
probability models as a basis for learning from the accruing data during
the trial, choosing doses for successive patient cohorts, and selecting
an MTD. These methods were illustrated and compared to the SM by
application to a particular trial in renal cell carcinoma. The paper
compared average behavior by computer simulation under each of several
hypothetical dose-toxicity curves. The comparisons showed that the
Bayesian methods are much more reliable than the conventional algorithms
for selecting an MTD, and that they have a low risk of treating patients
at unacceptably toxic doses. (Thall and Lee 2003)

Iasonos et al. (2008) compared several CRM-based methods with SM. For
comparison, variations were given to the number of dose levels (5 to 8)
and the location of the true MTD. Only CRM with constraint in dose
escalation was evaluated since it is more likely to be used by
clinicians as the O'Quigley et al.'s original CRM (O'Quigley, Pepe, and
Fisher 1990) allows skipping dose levels in the absence of DLT's,
potentially unnecessarily exposing patients to highly toxic drug levels
which makes clinicians uncomfortable. Furthermore, 3 CRM-based methods
that combine rule-based and model-based approaches were evaluated. They
found that CRM and SM are comparable in terms of how fast they reach the
MTD as well as the total sample size needed when testing a limited
number of dose levels (\(\leq\) 5), however, as the number of dose
levels was increased, CRM reached the MTD in fewer patients when used
with a fixed sample of 20 patients. However, a sample size of 20---25
patients is not sufficient to achieve a narrow precision around the
estimated toxicity rate at the MTD. CRM with a fixed-sample performed
better than a CRM with stopping rule that ensures a narrow confidence
interval around the toxicity rate at the MTD. CRM-based methods were
found to be better than SM in terms of accuracy and optimal dose
allocation in almost all cases except when the true dose was among the
lower levels. (Iasonos et al. 2008)

Onar et al.~in their paper (Onar, Kocak, and Boyett 2009) provided
modifications to CRM largely motivated by specific challenges
encountered in the context of the Pediatric Brain Tumor Consortium
trials and compared them to SM. While some versions of CRM assume
availability of doses in a continuous way given a range, this paper used
preset levels as it is more acceptable to clinicians and easier to
manage operationally especially in multi-institutional settings.
Patients in these pediatric trials were dosed by body surface area (BSA)
instead of in terms of ``mg'' as done in adult trials. A frequentist
likelihood-based approach with a 2-parameter logistic model (Piantadosi,
Fisher, and Grossman 1998) was used as follows: Phu(x\_j, a) = exp(alpha
+ beta* x\_j) / (1 + exp(alpha + beta* x\_j\}). Also, ``prior
information'' was used to fit the model which is needed especially
during early trial stages. Compared to SM, simulations indicated that
their modified CRM was more accurate, exposed fewer patients to
potentially toxic doses and tended to require fewer patients. They also
argued that as the CRM-based MTD has a consistent definition across
trials, it is convenient especially in consortium settings where
multiple agents are being tested in studies often running simultaneously
and accruing from the same patient population. (Onar, Kocak, and Boyett
2009)

Onar-Thomas et al. (Onar-Thomas and Xiong 2010) compared the performance
of CRM vs SM vs the Rolling-6 design via simulations. The Rolling-6
design is a relative newcomer developed with the intention to shorten
trial duration by minimizing the period of time during which the trial
is closed to accrual for toxicity assessment. Results indicated that the
toxicity rates were comparable across the 3 designs, but the SM and the
Rolling 6 designs tended to treat a higher \% of patients at doses below
the MTD. In cases where 5,6 or more dose levels were proposed to be
studied and some toxicities were expected, model-based designs (CRM) had
distinct advantages in being able to use the data from all dose levels
in estimating the MTD, in accommodating patient-specific dosing and in
providing an MTD estimate that is associated with a toxicity
probability. Doses identified as MTD by the SM and the Rolling-6
differed in a large \% of trials. Results also showed that body surface
area (BSA)-based dosing used in pediatric trials can make a difference
in dose escalation/de-escalation patterns in CRM relative to cases where
such variations are not taken into account in the calculations (such as
most adult trials) also leading to different MTDs in some cases.
Rolling-6 was found to have shorter trials relative to SM. CRM lead to
shorter trial duration for slow to medium accrual rates whereas
Rolling-6 may have an advantage if the accrual rate is fast. Rolling-6
may be preferable over the CRM if very few or no toxicity is expected
with the agent under study and if the dose finding period is long.
(Onar-Thomas and Xiong 2010)

\hypertarget{references-16-20-by-william}{%
\paragraph{2.2.4 References 16-20 (by
William)}\label{references-16-20-by-william}}

\hypertarget{reference-16}{%
\paragraph{Reference 16:}\label{reference-16}}

Le Tourneau et al. (2012)

This paper provide evidence that more extensive implementation of
innovative dose escalation designs such as mCRM and ATD in phase I
cancer clinical trials of molecularly targeted agents. This paper did a
literature review based on 84 trials that reached MTD. The goal was to
get more insight on the efficiency of new dose escalation methods in
phase I trials of molecularly targeted agents. The literature review
indicated a standard 3+3 design was used in 41 trials (49\%) while newer
algorithm based methods were also used, including ATD in 35 trials
(42\%) and CRM (mCRM), which was employed in only 6 trials (7\%). The
mean MTD to starting dose ratio appeared to be at least twice as high
for trials using a mCRM or an ATD as for trials using a standard ``3+3''
design. The mean number of patients exposed to a dose below the MTD for
all three trial designs was similar, ranging from 19 to 23. the mean
number of patients exposed to doses exceeding the MTD was at least twice
as high in trials using a standard ``3+3'' design or an ATD when
compared to trials using a mCRM

\hypertarget{reference-17}{%
\paragraph{Reference 17:}\label{reference-17}}

Garrett-Mayer (2006)

The paper `The continual reassessment method for dose-finding studies: a
tutorial' provides detailed explanation how to implement a CRM. Does
simulation with varying parameters to show how the CRM is better than
the 3+3

\hypertarget{reference-18-preview-version}{%
\paragraph{Reference 18 (Preview
version):}\label{reference-18-preview-version}}

Cheung (2011)

The book `Handbook of Methods for Designing, Monitoring, and Analyzing
Dose-Finding Trials' discusses Phase I designs, methodology for Phase
I/II. In addition various types of Phase II dose Finding trials are
discussed. This book is trying to understand the high failure rate in
Phase II clinical trials by improving the poor accuracy in early-phase
trials. Book reviews the classical 3+3 design and model based designs.
The book discusses barriers to adopting newer designs, available
software and tools, current usage of different designs in clinical
practice, modification of design rules related to start-up, stopping and
choice of endpoints. Model based designed discussed include the CRM and
escalation with overdose control

\hypertarget{reference-19-preview-version}{%
\paragraph{Reference 19 (Preview
version):}\label{reference-19-preview-version}}

O'Quigley, Iasonos, and Bornkamp (2017)

The book `Dose Finding by the Continual Reassessment Method' focusses on
CRM approach for studies designed with binary outcome and subjects come
from a homogenous population. This book is a `how-to' book to implement
CRM and includes what to do and what not to do. Book includes details on
how to calibrate CRM parameters based on general patterns. The book also
discusses the theoretical and methodological viewpoint including
pathological behaviors of CRM modification in simpler settings. The goal
of the book is finding dose criteria using a CRM design based on current
literature and the CRM trials the author helped implement.

\hypertarget{reference-20}{%
\paragraph{Reference 20:}\label{reference-20}}

Food, Administration, et al. (2013)

FDA guidance indicates Bayesian methods can be used in clinical trial
for Drugs and Biologics. This guidance indicates that Bayesian inference
is characterized by drawing conclusions based directly on posterior
probabilities that a drug is effective and has important differences
from frequentist inference (Berger and Wolpert 1988). For trials that
use Bayesian inference with informative prior distributions, such as
trials that explicitly borrow external information, Bayesian statistical
properties are more informative than Type I error probability.

\newpage

\hypertarget{methodology}{%
\section{3. Methodology}\label{methodology}}

\hypertarget{describe-the-use-of-the-logistic-regression-model-in-our-study-by-tongtong-jin}{%
\subsection{3.1 Describe the use of the logistic regression model in our
study (By Tongtong
Jin)}\label{describe-the-use-of-the-logistic-regression-model-in-our-study-by-tongtong-jin}}

The model we use here to predict the relation between dose level and the
probability of DLT is the two-parameter logistic regression model which
has the form as follows:
\[p_j=p(d_j|\beta_1,\beta_2)=\frac{exp(\beta_1+exp(\beta_2)d_j)}{1+exp(\beta_1+exp(\beta_2)d_j)}.\]
And in Bayesian setting, this is the likelihood function for dose level
j given \(\beta_1\) and \(\beta_2\).

\hypertarget{bayesian-crm}{%
\subsubsection{3.1.1 Bayesian CRM}\label{bayesian-crm}}

In the Bayesian setting of CRM, we first need to choose the prior
distributions of parameters \(\beta_1\) and \(\beta_2\). Let's denote
the prior by \(f(\beta_1,\beta_2)\). Then the posterior distribution
given the data of k dose levels \(D_k\) is as follows:
\[L(D_k|\beta_1,\beta_2)=\prod_{j=1}^{k}p_j^{y_j}(1-p_j)^{n_j-y_j} ,\]
where \(n_j\) is the number of tested patients at the j-th dose level,
and \(y_j\) is the number of patients showing DLT at the j-th dose
level. Then we can get to the posterior distribution given \(D_k\) by
applying above in the Bayes' rule. The posterior is
\[p_k(\beta_1,\beta_2|D_k)=\frac{L(D_k|\beta_1,\beta_2)f(\beta_1,\beta_2)}{\iint L(D_k|\beta_1,\beta_2)f(\beta_1,\beta_2)d \beta_1 d \beta_2}.\]

Then the posterior mean of DLT probability at each dose level is
\[\mathbf{E}[p_j|D_k]=\iint p_j p_k(\beta_1,\beta_2|D_k)d \beta_1 d \beta_2\]

To look for an appropriate dose level for the next trial, our principle
is to find the dose level with the DLT probability closest to TTL. Hence
the next dose level can be defined as
\[d_{next}=\arg\min_{d_j\in S}(|TTL-\mathbf{E}[p_j|D_k]|).\] Here S is
the set of all permissible choices of dose level.(Babb and Rogatko 2004)

\hypertarget{two-stage-likelihood-based-crm}{%
\subsubsection{3.1.2 Two-stage likelihood-based
CRM}\label{two-stage-likelihood-based-crm}}

Two-stage likelihood CRM divides the process into two stages. In the
first stage, the patients are dosed in single-patient cohorts until the
first DLT appears. After the first appearance of DLT, the CRM starts to
work on the data based on all the previous trials (first-stage data
included).

The stage 2 procedure is similar to the above, but using a maximum
likelihood estimation (MLE) to estimate the parameters \(\beta_1\) and
\(\beta_2\) and calculate the corresponding probability of DLT at each
dose level. The estimated parameters based on given data on k dose
levels are
\[(\hat{\beta_1},\hat{\beta_2})=\arg\max_{(\beta_1,\beta_2)} L(D_k|\beta_1,\beta_2).\]
Here \(L(D_k|\beta_1,\beta_2)\) is the same as defined above in the
Bayesian setting. Then we can compute the probability to DLT at each
dose level under the current MLE of the parameters
\(p(d_j|\hat{\beta_1},\hat{\beta_2})\). Now since we want the next dose
level to have the closest probability of DLT to TTL, we're able to
define the next dose level by
\[d_{next}=\arg\min_{d_j\in S}(|TTL-p(d_j|\hat{\beta_1},\hat{\beta_2})|).\]
Iterate above procedure until the dose level meets the stopping
condition. Then the second stage terminates.(Wages, Conaway, and
O'Quigley 2013)

\hypertarget{by-ujjwal-sehrawat}{%
\subsection{3.2 (By Ujjwal Sehrawat)}\label{by-ujjwal-sehrawat}}

\begin{itemize}
\tightlist
\item
  Note: content?
\end{itemize}

\hypertarget{implementation-plan-by-yiwei-ding}{%
\subsection{3.3 Implementation plan (By Yiwei
Ding)}\label{implementation-plan-by-yiwei-ding}}

Here we reproduce results in specific figures by fitting a logistic
regression model. We run simulations using the CRM with logistic
regression, and we derive Figure 6\&7. Figure 6 used a one-stage
Bayesian approach and Figure 7 used a two-stage likelihood-based
approach.

To reproduce Fig. 6, we should use a one-parameter logistic model and
placed an exponential prior distribution with a mean of 1. Throughout
the trial, calculations were made for the posterior estimates of the
likelihood of dose-limiting toxicity (DLT) at each dosage. The upcoming
group was then administered the dosage that had an estimated DLT
probability most closely aligning with the target toxicity level (TTL).

To reproduce Fig. 7, instead we implemente a two-stage likelihood-based
CRM design. We combined a one-parameter power model for the
dose-toxicity relationship. During the initial stage, virtual patients
were given gradually increasing doses, starting with a dose of 10 ng/kg,
which was 1\% of the Maximum Tolerable Dose (MTD) in dogs. If a grade 2+
non-DLT adverse event occurred in a patient, two more virtual patients
received the same dosage. If none of the trio exhibited a Dose-Limiting
Toxicity (DLT), the study continued escalating the dosage in the first
stage. Once the first DLT was observed, the second stage, based on the
model, was initiated.

A dose skeleton, which was determined after the first DLT (as it wasn't
needed during the first phase), was used to establish dose labels for
each dose. The likelihood of a DLT at each dose was determined using
maximum likelihood methods, and the upcoming patient was assigned the
dose that had an estimated DLT likelihood closest to the Target Toxicity
Level (TTL). This was under the condition that no dose level that hadn't
been tested could be bypassed. Single-patient cohorts were the norm
because a low toxicity incidence was anticipated, and each virtual
patient was thoroughly observed before the next patient was assigned a
dose.

\hypertarget{model-reproduction}{%
\section{Model Reproduction}\label{model-reproduction}}

\begin{itemize}
\tightlist
\item
  Reproduce results in Figures 6 \& 7
\end{itemize}

\begin{Shaded}
\begin{Highlighting}[]
\CommentTok{\# R code for model reproduction goes here}
\end{Highlighting}
\end{Shaded}

\hypertarget{simulation-studies}{%
\section{Simulation Studies}\label{simulation-studies}}

\begin{Shaded}
\begin{Highlighting}[]
\CommentTok{\# R code for simulation studies goes here}
\end{Highlighting}
\end{Shaded}

\hypertarget{results-and-discussion}{%
\section{Results and Discussion}\label{results-and-discussion}}

Answers to the questions

\hypertarget{conclusion}{%
\section{Conclusion}\label{conclusion}}

Summary of the report

\hypertarget{reference}{%
\section*{Reference}\label{reference}}
\addcontentsline{toc}{section}{Reference}

\hypertarget{refs}{}
\begin{CSLReferences}{1}{0}
\leavevmode\vadjust pre{\hypertarget{ref-AdaptiveDesignsWorkingGroup2017}{}}%
Adaptive Designs Working Group of the MRC Network of Hubs for Trials
Methodology Research. 2017. {``A Quick Guide Why Not to Use a+b
Designs.''} 2017.
\url{http://methodologyhubs.mrc.ac.uk/files/6814/6253/2385/A_quick_guide_why_not_to_use_AB_designs.pdf}.

\leavevmode\vadjust pre{\hypertarget{ref-babb2004bayesian}{}}%
Babb, James S, and André Rogatko. 2004. \emph{Bayesian Methods for
Cancer Phase i Clinical Trials}. New York: Marcel Dekker.

\leavevmode\vadjust pre{\hypertarget{ref-brock2017implementing}{}}%
Brock, Kristian, Lucinda Billingham, Mhairi Copland, Shamyla Siddique,
Mirjana Sirovica, and Christina Yap. 2017. {``Implementing the EffTox
Dose-Finding Design in the Matchpoint Trial.''} \emph{BMC Medical
Research Methodology} 17 (1): 1--15.

\leavevmode\vadjust pre{\hypertarget{ref-cater1972study}{}}%
Cater, SK. 1972. {``Study Design Principles for Clinical Evaluation of
New Drugs as Developed by the Chemotherapy Program of the National
Cancer Institute.''} \emph{The Design of Clinical Trials in Cancer
Therapy}, 242--89.

\leavevmode\vadjust pre{\hypertarget{ref-cheung2011dose}{}}%
Cheung, Ying Kuen. 2011. \emph{Dose Finding by the Continual
Reassessment Method}. CRC Press.

\leavevmode\vadjust pre{\hypertarget{ref-chevret1993continual}{}}%
Chevret, Sylvie. 1993. {``The Continual Reassessment Method in Cancer
Phase i Clinical Trials: A Simulation Study.''} \emph{Statistics in
Medicine} 12 (12): 1093--1108.

\leavevmode\vadjust pre{\hypertarget{ref-chiuzan2017dose}{}}%
Chiuzan, Cody, Jonathan Shtaynberger, Gulam A Manji, Jimmy K Duong, Gary
K Schwartz, Anastasia Ivanova, and Shing M Lee. 2017. {``Dose-Finding
Designs for Trials of Molecularly Targeted Agents and
Immunotherapies.''} \emph{Journal of Biopharmaceutical Statistics} 27
(3): 477--94.

\leavevmode\vadjust pre{\hypertarget{ref-food2013adaptive}{}}%
Food, Drug Administration, et al. 2013. {``Adaptive Design Clinical
Trials for Drugs and Biologics; 2018.''} \emph{Available at: Fda.
Gov/Regulatory-Information/Search-Fda-Guidance-Documents/Adaptive-Design-Clinical-Trials-Drugs-and-Biologics.
Accessed {[}September 12, 2019{]}}.

\leavevmode\vadjust pre{\hypertarget{ref-garrett2006continual}{}}%
Garrett-Mayer, Elizabeth. 2006. {``The Continual Reassessment Method for
Dose-Finding Studies: A Tutorial.''} \emph{Clinical Trials} 3 (1):
57--71.

\leavevmode\vadjust pre{\hypertarget{ref-harrington2013adaptive}{}}%
Harrington, Jennifer A, Graham M Wheeler, Michael J Sweeting, Adrian P
Mander, and Duncan I Jodrell. 2013. {``Adaptive Designs for Dual-Agent
Phase i Dose-Escalation Studies.''} \emph{Nature Reviews Clinical
Oncology} 10 (5): 277--88.

\leavevmode\vadjust pre{\hypertarget{ref-iasonos2016dimension}{}}%
Iasonos, Alexia, Nolan A Wages, Mark R Conaway, Ken Cheung, Ying Yuan,
and John O'Quigley. 2016. {``Dimension of Model Parameter Space and
Operating Characteristics in Adaptive Dose-Finding Studies.''}
\emph{Statistics in Medicine} 35 (21): 3760--75.

\leavevmode\vadjust pre{\hypertarget{ref-iasonos2008comprehensive}{}}%
Iasonos, Alexia, Andrew S Wilton, Elyn R Riedel, Venkatraman E Seshan,
and David R Spriggs. 2008. {``A Comprehensive Comparison of the
Continual Reassessment Method to the Standard 3+ 3 Dose Escalation
Scheme in Phase i Dose-Finding Studies.''} \emph{Clinical Trials} 5 (5):
465--77.

\leavevmode\vadjust pre{\hypertarget{ref-jaki2013principles}{}}%
Jaki, Thomas, Sally Clive, and Christopher J Weir. 2013. {``Principles
of Dose Finding Studies in Cancer: A Comparison of Trial Designs.''}
\emph{Cancer Chemotherapy and Pharmacology} 71: 1107--14.

\leavevmode\vadjust pre{\hypertarget{ref-korn1994comparison}{}}%
Korn, Edward L, Douglas Midthune, T Timothy Chen, Lawrence V Rubinstein,
Michaele C Christian, and Richard M Simon. 1994. {``A Comparison of Two
Phase i Trial Designs.''} \emph{Statistics in Medicine} 13 (18):
1799--1806.

\leavevmode\vadjust pre{\hypertarget{ref-kuen2013sample}{}}%
Kuen Cheung, Ying. 2013. {``Sample Size Formulae for the Bayesian
Continual Reassessment Method.''} \emph{Clinical Trials} 10 (6):
852--61.

\leavevmode\vadjust pre{\hypertarget{ref-le2012efficiency}{}}%
Le Tourneau, Christophe, Hui K Gan, Albiruni RA Razak, and Xavier
Paoletti. 2012. {``Efficiency of New Dose Escalation Designs in
Dose-Finding Phase i Trials of Molecularly Targeted Agents.''}
\emph{PloS One} 7 (12): e51039.

\leavevmode\vadjust pre{\hypertarget{ref-le2009dose}{}}%
Le Tourneau, Christophe, J Jack Lee, and Lillian L Siu. 2009. {``Dose
Escalation Methods in Phase i Cancer Clinical Trials.''} \emph{JNCI:
Journal of the National Cancer Institute} 101 (10): 708--20.

\leavevmode\vadjust pre{\hypertarget{ref-moller1995extension}{}}%
Møller, Susanne. 1995. {``An Extension of the Continual Reassessment
Methods Using a Preliminary up-and-down Design in a Dose Finding Study
in Cancer Patients, in Order to Investigate a Greater Range of Doses.''}
\emph{Statistics in Medicine} 14 (9): 911--22.

\leavevmode\vadjust pre{\hypertarget{ref-o1999another}{}}%
O'Quigley, John. 1999. {``Another Look at Two Phase i Clinical Trial
Designs.''} \emph{Statistics in Medicine} 18 (20): 2683--90.

\leavevmode\vadjust pre{\hypertarget{ref-o2017handbook}{}}%
O'Quigley, John, Alexia Iasonos, and Björn Bornkamp. 2017.
\emph{Handbook of Methods for Designing, Monitoring, and Analyzing
Dose-Finding Trials}. CRC press.

\leavevmode\vadjust pre{\hypertarget{ref-o1990continual}{}}%
O'Quigley, John, Margaret Pepe, and Lloyd Fisher. 1990. {``Continual
Reassessment Method: A Practical Design for Phase 1 Clinical Trials in
Cancer.''} \emph{Biometrics}, 33--48.

\leavevmode\vadjust pre{\hypertarget{ref-o2006experimental}{}}%
O'Quigley, John, and S Zohar. 2006. {``Experimental Designs for Phase i
and Phase i/II Dose-Finding Studies.''} \emph{British Journal of Cancer}
94 (5): 609--13.

\leavevmode\vadjust pre{\hypertarget{ref-onar2009continual}{}}%
Onar, Arzu, Mehmet Kocak, and James M Boyett. 2009. {``Continual
Reassessment Method Vs. Traditional Empirically Based Design:
Modifications Motivated by Phase i Trials in Pediatric Oncology by the
Pediatric Brain Tumor Consortium.''} \emph{Journal of Biopharmaceutical
Statistics} 19 (3): 437--55.

\leavevmode\vadjust pre{\hypertarget{ref-onar2010simulation}{}}%
Onar-Thomas, Arzu, and Zang Xiong. 2010. {``A Simulation-Based
Comparison of the Traditional Method, Rolling-6 Design and a Frequentist
Version of the Continual Reassessment Method with Special Attention to
Trial Duration in Pediatric Phase i Oncology Trials.''}
\emph{Contemporary Clinical Trials} 31 (3): 259--70.

\leavevmode\vadjust pre{\hypertarget{ref-penel2012does}{}}%
Penel, Nicolas, and Andrew Kramar. 2012. {``What Does a
Modified-Fibonacci Dose-Escalation Actually Correspond To?''} \emph{BMC
Medical Research Methodology} 12 (1): 1--5.

\leavevmode\vadjust pre{\hypertarget{ref-piantadosi1998practical}{}}%
Piantadosi, S, JD Fisher, and S Grossman. 1998. {``Practical
Implementation of a Modified Continual Reassessment Method for
Dose-Finding Trials.''} \emph{Cancer Chemotherapy and Pharmacology} 41:
429--36.

\leavevmode\vadjust pre{\hypertarget{ref-rogatko2007translation}{}}%
Rogatko, André, David Schoeneck, William Jonas, Mourad Tighiouart, Fadlo
R Khuri, and Alan Porter. 2007. {``Translation of Innovative Designs
into Phase i Trials.''} \emph{Journal of Clinical Oncology} 25 (31):
4982--86.

\leavevmode\vadjust pre{\hypertarget{ref-thall2003practical}{}}%
Thall, PF, and S-J Lee. 2003. {``Practical Model-Based Dose-Finding in
Phase i Clinical Trials: Methods Based on Toxicity.''}
\emph{International Journal of Gynecologic Cancer} 13 (3).

\leavevmode\vadjust pre{\hypertarget{ref-wages2013performance}{}}%
Wages, Nolan A, Mark R Conaway, and John O'Quigley. 2013. {``Performance
of Two-Stage Continual Reassessment Method Relative to an Optimal
Benchmark.''} \emph{Clinical Trials} 10 (6): 862--75.

\leavevmode\vadjust pre{\hypertarget{ref-wheeler2019design}{}}%
Wheeler, Graham M, Adrian P Mander, Alun Bedding, Kristian Brock,
Victoria Cornelius, Andrew P Grieve, Thomas Jaki, et al. 2019. {``How to
Design a Dose-Finding Study Using the Continual Reassessment Method.''}
\emph{BMC Medical Research Methodology} 19 (1): 1--15.

\end{CSLReferences}

\end{document}
